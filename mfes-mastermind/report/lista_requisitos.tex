
Inicialmente, foi-nos indicado no enunciado deste projeto que o
sistema deveria cumprir com estes 3 requisitos:

\begin{enumerate}
\item Perguntar ao sistema qual o número de peças com a cor certa e no
  sítio certo e qual o número de peças com a cor certa mas no sítio
  errado de cada tentativa;
\item Calcular o número de tentativas que o \\emph{codebreaker} precisou para
  adivinhar o código;
\item Guardar a informação sobre um campeonato com várias equipas e
  calcular a soma do número de tentativas que a equipa vencedora precisou
\end{enumerate}

A partir desses três requisitos pedidos e da descrição das regras do
Mastermind efetuada anteriormente, criámos um conjunto de requisitos
que o sistema tinha que implementar para funcionar corretamente:

\begin{enumerate}
\item Num jogo, com duas equipas, existem sempre duas partidas, em que
  cada equipa tenta quebrar o código definido pela adversária;
\item Dado um jogo, com duas equipas, deve ser possível a cada equipa
  definir o código a ser quebrado pela equipa adversária;
\item Em cada partida, cada equipa deve poder realizar uma tentativa
  de encontrar a solução do tabuleiro;
\item Dada uma tentativa, num determinado tabuleiro, deve ser possível
  saber quantas peças estão com a cor certa no lugar certo;
\item Dada uma tentativa, num determinado tabuleiro, deve ser possível
  saber quantas peças estão com a cor certa no lugar errado;
\item Após uma tentativa, deve ser possível verificar se a solução foi
  encontrada;
\item Deve ser possível calcular o número de tentativas que foram
  necessárias para adivinhar a solução;
\item Deve ser possível criar um campeonato, onde participem várias
  equipas;
\item Deve ser possível adicionar equipas ao campeonato;
\item As equipas que participam num determinado jogo de um determinado
  campeonato, têm que estar inscritas no campeonato;
\item Dada uma equipa, deve ser possível obter a sua pontuação num
  dado campeonato. Esse valor depende do número de tentativas que as
  equipas adversárias gastaram para adivinhar a solução;
\item Um campeonato tem que ter no minimo 2 equipas;
\item Deve ser possível gravar um campeonato para um ficheiro e
  restaurá-lo mais tarde;
\item Deve ser possível adicionar jogos aos campeonatos.
\end{enumerate}

\subsection{Restrições necessárias para o funcionamento correto do sistema}

Foram também identificadas um conjunto de restrições que o sistema
tinha que cumprir para funcionar corretamente:

\begin{enumerate}
\item Após a solução de um tabuleiro ter sido encontrada, não é possível fazer mais
  nenhuma jogada nele;
\item Só é possível efetuar uma tentativa se o jogo ainda não tiver terminado;
\item As cores usadas nos códigos têm que pertencer a um conjunto de 6
  cores definidos previamente;
\item Cada tabuleiro só aceita tentativas e soluções com 4 elementos;
\item O número máximo de tentativas de cada tabuleiro é 8, 10, ou 12;
\item O número de peças com a cor certa no sítio certo não pode ser
  superior ao tamanho da solução do tabuleiro;
\item O número de peças com a cor certa, mas no sítio errado não pode
  ser superior ao tamanho da solução do tabuleiro;
\item O número de equipas num campeonato tem de ser um número par
  maior ou igual a 2;
\item As equipas que participam nos jogos de um campeonato têm que
  estar inscritas nesse campeonato;
\end{enumerate}

