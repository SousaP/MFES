Este relatório foi realizado no âmbito da unidade curricular de
Métodos Formais Formais em Engenharia de Software do 4º ano do
Mestrado Integrado em Engenharia Informática e Computação da Faculdade
de Engenharia da Universidade do Porto.

Este documento especifica, utilizando a linguagem VDM++, um modelo
formal das regras do conhecido jogo Mastermind, pretendendo assim
dotar os elementos do grupo de capacidades que lhes permitam criar no
futuro modelos formais que permitam representar e interagir com outros
sistemas.

Este relatório terá a seguinte estrutura: primeiro iremos descrever as
regras do jogo Mastermind, sendo depois enumerados os requisitos e
restrições que foram implementados neste modelo.

Após essa descrição será demonstrado como é que as restrições foram
implementadas no modelo escrito em VDM++.

De seguida iremos explicar o método usado para escrever e ler objetos
a partir do disco, e quais as razões pelas quais esta funcionalidade
não está implementada no código Java gerado.

Iremos depois descrever a bateria de testes utilizada para testar o
modelo, e será mostrada tanto a matriz de rastreabilidade como o
diagrama de classes gerado pelo Enterprise Architect.

Será depois colocado o código completo das classes utilizadas neste
trabalho bem como a sua cobertura por parte dos testes definidos
anteriormente.

Por fim será descrito a funcionalidade de análise de consistência do
modelo disponível nas ferramentas utilizadas para desenvolver este
modelo.

Na realização deste trabalho foram gastas aproximadamente 60 horas para especificar o
modelo, 2 horas para construir e executar os casos de teste e 10 horas
para gerar o código adicional necessário para fornecer uma
implementação a funcionar
