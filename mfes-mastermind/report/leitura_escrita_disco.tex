Uma dos requisitos pedidos pelo enunciado deste trabalho era a
capacidade de poder gravar campeonatos para o disco e restaurá-los a
partir do disco.

Infelizmente o código que implementa esse requisito só funciona quando
corrido na interface VDMTools ou Overture, pois é convertido para
código Java que não é válido, devido ao fato de este não conter um
sistema de funções anónimas (também conhecidas por ``lambdas'').

Esta secção serve então para explicar melhor o sistema criado para
implementar esse sistema.

\subsection{A teoria}

Em primeiro lugar, é necessário recorrer à biblioteca standard
\texttt{IO.vdmpp} do VDM que implementa o sistema de ``Input'' e
``Output'' usado pela linguagem VDM, pelo que esta deverá estar
importada no projeto.

De seguida, iremos explicar como funciona o sistema de ``Input'' e
``Output'' do VDM.

Consideremos a seguinte classe escrita em VDM++:

\begin{vdm_al}
  class Foo
    instance variables
    bar : seq of Char;
    
    operations
    Foo : nat ==> Foo
    Foo : (quaz) ==
      bar := quaz;
  
    getBar : () ==> nat
    getBar () == return bar;
  end Foo
\end{vdm_al}

Se no interpretador do VDMTools escrevermos:

\begin{verbatim}
  p new Foo("Hello, World!")
\end{verbatim}

O output será:

\begin{verbatim}
  objref3(Foo):
  <  - Foo`bar = "Hello, World!" >
\end{verbatim}

Portanto, é criado um objecto do tipo ``Foo'' com o valor da variável
``bar'' a 5.

Sabendo que a função \texttt{IO`freadval[@T](seq of char)} lê o conteúdo
de um ficheiro cujo nome é passado como seu argumento e passa esse
conteúdo para o interpretador do VDMTools, então poderemos recriar
este objecto desde que consigamos colocar num ficheiro a seguinte
``string'':

\begin{verbatim}
  new Foo("Hello, World!")
\end{verbatim}

Uma forma simples de o fazer seria criar um operador
``dumpFooToString'' que devolvesse uma ``string'' que permitisse
recriar o objecto. Isto é feito usando o seguinte operador:

\begin{vdm_al}
  dumpFooToString : () ==> seq of char
  dumpFooToString () ==
    return "new Foo(\"" ^ bar ^ "\")"
\end{vdm_al}

Após adicionar este operador à classe ``Foo'' torna-se possível
efetuar o seguinte no interpretador:

\begin{verbatim}
  p new Foo("Hello, World!").dumpFooToString()
\end{verbatim}

Que retorna a seguinte ``string'':

\begin{verbatim}
  "new Foo(\"Hello, World!\")"
\end{verbatim}

Se se tentasse escrever essa ``string'' para um ficheiro (através do
uso
do operador \texttt{IO`fwriteval}) o operador
\texttt{IO`freadeval} não o conseguiria ler devido ao fato de os
caracteres \texttt{"} estarem escritos assim: \verb=\"= .

Felizmente, a biblioteca ``IO'' disponibiliza um operador chamado
\texttt{IO`fecho} que escreve para um ficheiro uma ``string'', mas
removendo os caracteres \verb=\=.

Portanto para poder gravar uma classe para o disco e depois voltar a
ler-la temos que realizar os seguintes passos:

\begin{enumerate}
\item Adicionar a biblioteca \texttt{IO.vdmpp} ao projeto;
\item Adicionar um construtor à classe que permita repor os valores de
  todas as suas variáveis internas;
\item Adicionar um operador à classe que permita escrever uma ``string''
  no formato ``new Foo(arg1, arg2, \ldots, argN)'';
\item Escrever essa ``string'' para um ficheiro, utilizando o operador
  \texttt{IO`fecho};
\item Ler de volta essa ``string'' utilizando o operador
  \texttt{IO`freadval}.
\end{enumerate}

\subsection{A prática}

No exemplo mostrado anteriormente a criação do operador
``dumpFooToString'' era simples devido ao fato de requerer
simplesmente uma concatenação de três sequências de caracteres: a
sequência \texttt{new Foo("} seria concatenada com a sequência
guardada em ``bar'' que por sua vez seria concatenada com a sequência
\texttt{")} dando assim origem à ``string'' necessária.

Infelizmente o sistema implementado neste relatório não contêm só
variáveis do tipo ``String'', mas também funções finitas e
conjuntos. Será por isso implementar funções que convertam esses tipos
de variáveis para ``String''.

Para isso foram implementadas um conjunto de funções polimórficas na
classe ``Utilities''.

A função ``seqToString'' recebe uma sequência de variáveis de um dado
tipo e uma função que ao receber uma variável desse tipo a converte
para uma ``string'' que a representa.

\begin{vdm_al}
  static public seqToString[@T] : seq of [@T] * (@T -> String) -> String
  seqToString (sequence, printer) ==
    if sequence = [] then "[]"
    else "[" ^ auxSeqToString[@T](sequence, printer) ^ "]";
\end{vdm_al}

No entanto a função ``seqToString'' não é mais do que uma simples
interface para a função ``auxSeqToString'', com a particularidade de
tratar da situação em que o argumento ``sequence'' é vazio.

A função ``auxSeqToString'' é simplesmente uma função recursiva que
itera pelos elementos existentes na sequência passada pelo argumento
``sequence'' e armazena o valor de retorno da função ``printer''
quando chamada com esse elemento. Esses valores de retorno são
separados por \texttt{", "}.

\begin{vdm_al}
  static private auxSeqToString[@T] : seq1 of [@T] * (@T -> String) -> String
  auxSeqToString (sequence, printer) ==
    if len sequence = 1 then
      printer(hd sequence)
    else
      printer(hd sequence) ^ ", " ^ auxSeqToString[@T](tl sequence, printer);
\end{vdm_al}

No entanto não é necessário implementar só a conversão de sequências
para ``strings'', sendo também necessário implementar a conversão de
conjuntos para ``strings''.

Para isso foi escrito também a função ``setToString'' que reaproveita
a função ``auxSeqToString'' para converter um set numa ``string''.

\begin{vdm_al}
  static public setToString[@T] : set of @T * (@T -> String) -> String
  setToString(s, printer) ==
    if s = {} then "{}"
    else "{" ^ auxSeqToString[@T](setToSeq[@T](s), printer) ^ "}";
\end{vdm_al}

Como se pode ver, a função ``setToString'' é semelhante à função
``seqToString'': ambas recebem um argumento para converter e uma
função que imprime as variáveis desse argumento para uma ``string'',
ambas tratam do caso em que o conjunto é vazio e ambas remetem o
trabalho para uma função auxiliar.

No entanto a função ``setToString'' tem uma particularidade: em vez de
implementar uma função auxiliar ``auxSetToString'', esta chama a
função ``auxSeqToString'' depois de converter o conjunto que recebe no
argumento ``s'' para uma sequência.

Esta conversão é efetuada através da função ``setToSeq'' que
simplesmente extrai cada elemento do conjunto individualmente e
coloca-os em sequências que depois são concatenadas umas com as outras
para dar resultado a uma sequência que contêm todos os elementos dos
conjunto.

Como a ordem dos elementos de um conjunto não é importante, a ordem
pela qual são convertidos para uma sequência também não o é.

Tendo agora estas funções definidas torna-se mais simples de escrever
os objetos das classes no disco.

Começaremos então pela classe mais simples: a classe ``Board''.

A classe ``Board'' contêm três variáveis internas:

\begin{itemize}
\item A variável ``solution'' do tipo \texttt{seq1 of Color`Color};
\item A variável ``attempts'' do tipo \texttt{seq of (seq1 of Color`Color)};
\item A variável ``maxAttempts'' do tipo \texttt{nat1}.
\end{itemize}

É por isso necessário adicionar o seguinte construtor à classe
``Board'':

\begin{vdm_al}
  -- Constructor needed to recreate a board from the information of a file
  public Board : seq1 of Color`Color * nat1 * seq of (seq1 of Color`Color) ==> Board
  Board (correctPlay, maxNumberOfTries, savedAttempts) == (
    solution := correctPlay;
    maxAttempts := maxNumberOfTries;
    attempts := savedAttempts;
  )
  pre len correctPlay = attemptLength and
    maxNumberOfTries in set maxAttemptsAvailable and
    len savedAttempts <= maxNumberOfTries and
    forall attempt in set elems savedAttempts & len attempt = attemptLength

  post solution = correctPlay and
    maxAttempts = maxNumberOfTries and
    -- All elements of "attempts" need to be in the same position of the
    -- elements of "savedAttempts"
    numberOfMatchingElems[seq1 of Color`Color](attempts, savedAttempts) = len attempts;
\end{vdm_al}

Para que ninguém possa alterar os dados que estão no ficheiro de forma
a quebrar as restrições da classe ``Board'', este construtor também
contêm pré e pós condições que restringem os dados que ele pode receber.

O operador ``dumpBoardToString'' é o operador responsável por criar
uma ``string'' com o seguinte formato:

\begin{verbatim}
  new Board(<solution>, <maxAttempts>, <attempts>)
\end{verbatim}

Como se pode ver, este operador utiliza dois operadores auxiliares: o
``maxAttemptsToString'' e o ``attemptsToString'', que convertem,
respetivamente as variáveis ``maxAttempts'' e ``attempts'' para uma
``string''. De notar que como a variável ``solution'' já é do tipo
``string'' esta não precisa de ser convertida.

\begin{vdm_al}
  public dumpBoardToString : () ==> String
  dumpBoardToString () ==
    return "new Board(\"" ^ solution ^ "\", " ^ maxAttemptsToString()
      ^ ", " ^ attemptsToString() ^")";
\end{vdm_al}

O operador ``maxAttemptsToString'' é simplesmente uma expressão do
tipo ``if'' que devolve uma representação em ``string'' do valor que a
variável ``maxAttempts'' tem neste tabuleiro.


\begin{vdm_al}
  private maxAttemptsToString : () ==> String
  maxAttemptsToString () ==
    if maxAttempts = 8 then return "8"
    elseif maxAttempts = 10 then return "10"
    else return "12"
  pre maxAttempts = 8 or maxAttempts = 10 or maxAttempts = 12;
\end{vdm_al}

Por outro lado o operador ``attemptsToString'' utiliza a função
``seqToString'' já mostrada.

\begin{vdm_al}
  private attemptsToString : () ==> String
  attemptsToString () ==
    return Utilities`seqToString[seq1 of Color`Color](attempts,
      lambda x : String & "\"" ^ x ^ "\"");
\end{vdm_al}

De seguida iremos analisar a classe ``Game''.

Esta classe só tem uma variável interna: a variável ``gameInstances''
que é uma variável do tipo \texttt{map String to Board}.

Por isso o construtor a adicionar à classe ``Game'' apenas recebe um
argumento.


\begin{vdm_al}
  -- Constructor needed to recreate a game from the information of a file
  public Game : map String to Board ==> Game
  Game (storedGameInstances) ==
    gameInstances := storedGameInstances
  pre card (dom storedGameInstances) = 2
  post gameInstances = storedGameInstances;
\end{vdm_al}

Tal como aconteceu com a classe ``Board'' este construtor garante que
a informação lida do disco cumpre com as restrições definidas para as
variáveis da classe ``Game'', com uma exceção: como a classe ``Game''
não guarda em nenhuma variável a informação sobre o campeonato a que
pertence, este construtor não tem forma de saber se as equipas que
recebe pertencem de fato ao campeonato. Como será visto mais à frente,
é da responsabilidade da classe ``Championship'' garantir que as
equipas estão inscritas em si quando esta cria os objetos da classe
``Game'' a partir de um ficheiro.

Tal como já tinha acontecido com a classe ``Board'', também a classe
``Game'' contêm uma operação ``dumpGameToString''.

\begin{vdm_al}
  public dumpGameToString : () ==> String
  dumpGameToString () == return "new Game(" ^ dumpGameInstanceToString() ^ ")";
\end{vdm_al}

Felizmente conhece-se \emph{à priori} a estrutura da variável
``gameInstances'', por isso pode-se escrever o operador
``dumpGameInstanceToString'' sem ter que utilizar funções semelhantes
a ``setToString'' ou ``seqToString''.

É também importante reparar que neste operador é chamado o operador
``dumpBoardToString'' definido anteriormente que converte um objeto do
tipo ``Board'' para uma ``string'' que pode ser posterior lida pelo
interpretador do VDM.

\begin{vdm_al}
  private dumpGameInstanceToString : () ==> String
  dumpGameInstanceToString () ==
    let {team1, team2} = dom gameInstances in
    return "{\"" ^ team1 ^ "\" |-> " ^ gameInstances(team1).dumpBoardToString() ^ ", "
      ^ "\"" ^ team2 ^ "\" |-> " ^ gameInstances(team2).dumpBoardToString() ^ "}"
  pre card (dom gameInstances) = 2;
\end{vdm_al}

Finalmente a classe ``Championship'' contêm duas variáveis internas:

\begin{itemize}
\item A variável ``teams'' do tipo \texttt{set of String};
\item A variável ``games'' do tipo \texttt{set of Game}.
\end{itemize}

Portanto o construtor que irá recriar um campeonato a partir do
ficheiro será este:

\begin{vdm_al}
  -- Constructor needed to recreate a championship from the information of
  -- a file
  public Championship: set of String * set of Game ==> Championship
  Championship(participants, gamesPlayed) == (
    teams := participants;
    games := gamesPlayed
  )
  pre (card participants) >= 2 and (card participants) mod 2 = 0 and
    forall game in set gamesPlayed & game.getParticipantTeams() subset participants
  post teams = participants and games = gamesPlayed;
\end{vdm_al}

De notar que a restrição que o construtor ``Game'' mencionado
anteriormente não podia garantir (as equipas que participam no jogo
estão inscritas no campeonato) é verificada aqui graças à pré-condição
deste construtor. Recordar que, para além deste construtor, a variável
``games'' da classe ``Championship'' também contêm uma invariante
sobre ela que garante que todas as equipas que estão nos jogos aí
guardados estão inscritos no campeonato.

Tal como acontecia com as classes ``Board'' e ``Game'', a conversão de
um campeonato requer a criação de um operador
``dumpChampionshipToString''.


\begin{vdm_al}
  public dumpChampionshipToString : () ==> String
  dumpChampionshipToString() ==
    return "new Championship(" ^ teamsToString() ^ ", " ^ gamesToString() ^ ")";
\end{vdm_al}

O operador ``teamsToString'' simplesmente chama a função
``setToString'' mostrada anterior para converter a variável ``teams''
para uma ``string''.

\begin{vdm_al}
  private teamsToString : () ==> String
    teamsToString () ==
      return Utilities`setToString[String](teams,
        lambda x : String & "\"" ^ x ^ "\"");
\end{vdm_al}

Seria de esperar que o operador ``gamesToString'' pudesse ser
implementado de forma análoga ao operador ``teams\-ToString'', ou seja,
com uma implementação que usasse a função ``setToString''.

\begin{vdm_al}
  private gamesToString : () ==> String
  gamesToString () ==
    return Utilities`setToString[Game](games, lambda x : Game & x.dumpGameToString());
\end{vdm_al}

E enquanto que esta definição funciona no VDMTools, quando ela é
testada no software Overture, este dá o seguinte erro no seu
\emph{engine}:

\begin{verbatim}
  Overture crashes with the following error:
  Illegal clone: java.lang.NullPointerException

  Main 206: Error evaluating code
  Detailed Message: Illegal clone: java.lang.NullPointerException
\end{verbatim}

Por essa razão foi necessário implementar o operador ``gamesToString''
num estilo de programação imperativo.

\begin{vdm_al}
  private gamesToString : () ==> String
  gamesToString () ==
    if card games = 0 then return "{}"
    else (
      dcl return_value : String := "{",
        i : nat1 := 1;
      for all game in set games do
        if i < card games then (
          return_value := return_value  ^ " " ^ game.dumpGameToString() ^ ", ";
          i := i + 1;
        )
        else
          return_value := return_value ^ " " ^ game.dumpGameToString() ^ "}";

      return return_value;
   );
\end{vdm_al}

É também devido a este erro no Overture que a função ``setToString''
não aparece total coberta nos testes, pois enquanto que existem testes
que de fato testam a condição de chamar a função ``setToString'' com
um conjunto vazio, estes provocam erros no Overture, que é a
ferramenta usada neste trabalho para gerar a cobertura de código. 


Como demonstração desta funcionalidade de escrita e leitura de objetos
para o disco, fica aqui um registo de uma sessão do utilizador com o
interpretador do VDMTools onde é mostrada a escrita de um campeonato
para o disco e a sua leitura.

De notar que no final, os campeonatos nas variáveis ``c'' e ``f'' são
idênticos.

O texto a negrito corresponde a dados introduzidos pelo utilizador no
interpretador do VDMTools.

\begin{Verbatim}[commandchars=+*/]
Initializing specification ... 
done
>> +textbf*create c := new Championship({"lorem", "ipsum"})/
>> +textbf*print c.addGame({"lorem" |-> "bbrr", "ipsum" |-> "gmgg"})/
objref12(Game):
  <  - Game`gameInstances = { "ipsum" |-> 
        objref14(Board):
          <  - Board`attempts = [  ],
             - Board`solution = "bbrr",
             - Board`maxAttempts = 12 >
,
      "lorem" |-> 
        objref13(Board):
          <  - Board`attempts = [  ],
             - Board`solution = "gmgg",
             - Board`maxAttempts = 12 >
 } >

>> +textbf*print c/
objref11(Championship):
  <  - Championship`games = { objref12(Game):
        <  - Game`gameInstances = { "ipsum" |-> 
              objref14(Board):
                <  - Board`attempts = [  ],
                   - Board`solution = "bbrr",
                   - Board`maxAttempts = 12 >
,
            "lorem" |-> 
              objref13(Board):
                <  - Board`attempts = [  ],
                   - Board`solution = "gmgg",
                   - Board`maxAttempts = 12 >
 } >
 },
     - Championship`teams = { "ipsum", "lorem" } >

>> +textbf*print IO`fecho("champ1.txt", c.dumpChampionshipToString(), <start>)/
true
>> +textbf*create f := IO`freadval[Championship]("champ1.txt").#2/
>> +textbf*print f/
objref20(Championship):
  <  - Championship`games = { objref19(Game):
        <  - Game`gameInstances = { "ipsum" |-> 
              objref18(Board):
                <  - Board`attempts = [  ],
                   - Board`solution = "bbrr",
                   - Board`maxAttempts = 12 >
,
            "lorem" |-> 
              objref17(Board):
                <  - Board`attempts = [  ],
                   - Board`solution = "gmgg",
                   - Board`maxAttempts = 12 >
 } >
 },
     - Championship`teams = { "ipsum", "lorem" } >
\end{Verbatim}

% No final, o disco contêm um ficheiro ``champ1.txt'' com o seguinte
% conteúdo:


