\begin{vdm_al}
class Championship

 types
  public String = Utilities`String;

 instance variables
  private teams : set of String;
  inv card teams >= 2 and (card teams) mod 2 = 0;

  private games : set of Game := {};
  inv forall game in set games & game.getParticipantTeams() subset teams;


 operations

  public Championship: set of String ==> Championship
  Championship(participants) == (

   teams := participants;

  )
  pre card participants >= 2 and (card participants) mod 2 = 0
  post teams = participants;


  -- Constructor needed to recreate a championship from the information of
  -- a file
  public Championship: set of String * set of Game ==> Championship
  Championship(participants, gamesPlayed) == (
   teams := participants;
   games := gamesPlayed
  )
  pre (card participants) >= 2 and (card participants) mod 2 = 0 and
   forall game in set gamesPlayed & game.getParticipantTeams() subset participants
  post teams = participants and games = gamesPlayed;


  public getTeams: () ==> set of String
  getTeams() == (

   return teams;

  )
  post RESULT = teams;


  public getGames: () ==> set of Game
  getGames() == (

   return games;

  )
  post RESULT = games;


  public addQuickGame: String * String ==> Game
  addQuickGame(team1, team2) == (

   dcl g : Game := new Game(
    {team1 |-> Color`getRandomSequence(Board`attemptLength),
     team2 |-> Color`getRandomSequence(Board`attemptLength)},
    self);

   games := games union {g};

   return g;

  )
  pre {team1, team2} subset teams and
   {team1, team2} not in set oldMatchups()
  post RESULT in set games;


  public addGame : map String to (seq of Color`Color) ==> Game
  addGame(teamsSolutions) == (

   dcl g : Game := new Game(teamsSolutions, self);

   games := games union {g};

   return g;

  )
  pre dom teamsSolutions subset teams and
   dom teamsSolutions not in set oldMatchups() and
   forall solution in set rng teamsSolutions & len solution = Board`attemptLength
  post RESULT in set games;


  public addTeams : set of String ==> ()
  addTeams(newTeams) == (

   teams := teams union newTeams;

  )
  pre (newTeams inter teams = {}) and (card newTeams) mod 2 = 0
  post teams = teams~ union newTeams;

  -- The codemaker gets one point for each guess a codebreaker makes. An
  -- extra point is earned by the codemaker if the codebreaker doesn't
  -- guess the pattern exactly in the last guess.
  public getScore : String ==> nat
  getScore (team) == (
   dcl gamesPlayed : set of Game := gamesPlayedByTeam(team),
    scoreByWinning : nat := 0,
    scoreByUndefeated : nat := 0;

    for board in getBoardsCrackedByOpponent(gamesPlayed, team) do
     scoreByWinning := scoreByWinning + board.numberOfTriesMade();

    for board in getBoardsUncrackedByOpponent(gamesPlayed, team) do
     scoreByUndefeated := scoreByUndefeated + board.numberOfTriesMade()+1;

    return scoreByWinning + scoreByUndefeated;

  )
  pre team in set teams;

  private oldMatchups : () ==> set of set of String
  oldMatchups () ==
   return {oldGames.getParticipantTeams() | oldGames in set games}
  post games = {} => RESULT = {};


  public gamesPlayedByTeam : String ==> set of Game
  gamesPlayedByTeam (team) ==
   return {game | game in set games & team in set game.getParticipantTeams()}
  post forall game in set RESULT
   & team in set game.getParticipantTeams();


  public getNumberOfTriesWinner : () ==> nat
  getNumberOfTriesWinner () ==
   let scoreBoard = getScoreboard()
   in
    return maximumValue(rng scoreBoard)
  pre let scoreboard = getScoreboard()
   in card dom (scoreboard :> {maximumValue(rng scoreboard)}) = 1;

  public getScoreboard : () ==> map String to nat
  getScoreboard () ==
   return { team |-> getScore(team) | team in set teams };


  -- Operators needed to write a Championship to a file
  private teamsToString : () ==> String
  teamsToString () ==
   return Utilities`setToString[String](teams,
    lambda x : String & "\"" ^ x ^ "\"");

  -- private gamesToString : () ==> String
  -- gamesToString () ==
  -- return Utilities`setToString[Game](games,
  --  lambda x : Game & x.dumpGameToString());
  --
  -- I wanted to implement "gamesToString" like this, but while it works
  -- with VDMTools, it does not work with Overture for some strange reason.
  -- Overture crashes with the following error:
  -- Illegal clone: java.lang.NullPointerException
  --
  -- Main 206: Error evaluating code
  -- Detailed Message: Illegal clone: java.lang.NullPointerException
  --
  -- So I needed to use the ugly implementation below, because it works in
  -- both systems.
  --  - Rolando

  private gamesToString : () ==> String
  gamesToString () ==
   if card games = 0 then return "{}"
   else (
    dcl return_value : String := "{",
     i : nat1 := 1;
    for all game in set games do
     if i < card games then (
      return_value := return_value  ^ " " ^ game.dumpGameToString() ^ ", ";
      i := i + 1;
     )
     else
      return_value := return_value ^ " " ^ game.dumpGameToString() ^ "}";

    return return_value;
   );

  public dumpChampionshipToString : () ==> String
  dumpChampionshipToString() ==
   return "new Championship(" ^ teamsToString() ^ ", " ^ gamesToString() ^ ")";


  functions

  private getBoardsUncrackedByOpponent : set of Game * String -> seq of Board
  getBoardsUncrackedByOpponent (games, team) ==
   if games = {} then []
   else let game in set games in
    let boardPlayedByOpponent = game.getBoardPlayedByOpponent(team) in
     if boardPlayedByOpponent.isGameOver() and
      not boardPlayedByOpponent.isSolutionCracked() then
       [boardPlayedByOpponent] ^ getBoardsUncrackedByOpponent(games\{game}, team)
     else
      getBoardsUncrackedByOpponent(games\{game}, team)

  post forall board in set elems RESULT & board.isGameOver() and not board.isSolutionCracked();


  private getBoardsCrackedByOpponent : set of Game * String -> seq of Board
  getBoardsCrackedByOpponent (games, team) ==
   if games = {} then []
   else let game in set games in
    let boardPlayedByOpponent = game.getBoardPlayedByOpponent(team) in
     if boardPlayedByOpponent.isGameOver() and
      boardPlayedByOpponent.isSolutionCracked() then
       [boardPlayedByOpponent] ^ getBoardsCrackedByOpponent(games\{game}, team)
     else
      getBoardsCrackedByOpponent(games\{game}, team)

  post forall board in set elems RESULT & board.isGameOver() and board.isSolutionCracked();

  
  private maximumValue : set of nat -> nat
  maximumValue (s) ==
   if card s = 1 then
    let x in set s in x
   else
    let x in set s in
     let max = maximumValue(s\{x}) in
      if x > max then
       (*@\notcovered{x}@*)
      else
       max
  pre s <> {}
  post not exists element in set s & element > RESULT;

 
end Championship
\end{vdm_al}
\bigskip
\begin{longtable}{|l|r|r|}
\hline
Function or operation & Coverage & Calls \\
\hline
\hline
Championship & 100.0\% & 5 \\
\hline
addGame & 100.0\% & 8 \\
\hline
addQuickGame & 100.0\% & 1 \\
\hline
addTeams & 100.0\% & 1 \\
\hline
dumpChampionshipToString & 100.0\% & 5 \\
\hline
gamesPlayedByTeam & 100.0\% & 16 \\
\hline
gamesToString & 100.0\% & 5 \\
\hline
getBoardsCrackedByOpponent & 100.0\% & 30 \\
\hline
getBoardsUncrackedByOpponent & 100.0\% & 30 \\
\hline
getGames & 100.0\% & 21 \\
\hline
getNumberOfTriesWinner & 100.0\% & 1 \\
\hline
getScore & 100.0\% & 16 \\
\hline
getScoreboard & 100.0\% & 2 \\
\hline
getTeams & 100.0\% & 36 \\
\hline
maximumValue & 96.7\% & 6 \\
\hline
oldMatchups & 100.0\% & 9 \\
\hline
teamsToString & 100.0\% & 5 \\
\hline
\hline
Championship.vdmpp & 99.7\% & 197 \\
\hline
\end{longtable}

