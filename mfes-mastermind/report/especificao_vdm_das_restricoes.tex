
Tendo enumerado agora as restrições que o sistema deve cumprir para
funcionar corretamente, iremos agora explicar como estas foram
implementadas no modelo de VDM.

A primeira restrição, que indica que a após a solução de um tabuleiro
ter sido encontrada não é possível fazer mais nenhuma jogada nele foi
implementada através da pré-condição implementada no operador
``makeAPlay'' da classe Board.

\begin{vdm_al}
  public makeAPlay : seq1 of Color`Color ==> ()
  makeAPlay (attempt) == ...
  pre len attempt = attemptLength and
      not isGameOver()
  post attempts = attempts~ ^ [attempt];
\end{vdm_al}

Como se pode ver pela pós-condição do operador ``isGameOver'' se a
solução já tiver sido encontrada, então este operador irá
garantidamente devolver ``true'' o que fazer com que a pré-condição do
operador ``makeAPlay'' seja falsa.

\begin{vdm_al}
  public isGameOver : () ==> bool
    isGameOver () == ...
  post RESULT = (isSolutionCracked() or len attempts = maxAttempts);
\end{vdm_al}

O operador ``isSolutionCracked'' verifica simplesmente se a solução
está na sequência de tentativas realizadas.

\begin{vdm_al}
  -- Return true if the correct code has been found
  public isSolutionCracked : () ==> bool
    isSolutionCracked () == return solution in set elems attempts
  post solution in set elems attempts => RESULT = true;
\end{vdm_al}

Como foi mostrado anteriormente, a definição de ``isGameOver'' devolve
o valor ``true'' quando o tamanho da sequência de tentativas for igual
ao número máximo de tentativas definidas para aquele tabuleiro (valor
dado pela variável ``maxAttempts''). Assim, a restrição 2 ``Só é
possível efetuar uma tentativa se o jogo ainda não tiver terminado'' é
também implementada com o uso a uma pré-condição no operador
``makeAPlay''.

Para implementar a restrição 3, que indica que uma cor usada num
tabuleiro tem que pertencer a um conjunto de cores previamente
definido foi criada uma invariante sobre o tipo Color definido na
class Color, que indica que cada variável do tipo Color têm que ser
uma das seguintes cores: ``b''lue, ``g''reen, ``r''ed, ``o''range,
``y''ellow, ``m''agenta.

\begin{vdm_al}
  types
    public Color = char
    inv color == color in set {'b', 'g', 'r', 'o', 'y', 'm'};
\end{vdm_al}

Para implementar a restrição 4, que limita o tamanho das soluções e
tentativas feitas num tabuleiro a sequências de 4 cores foram escritas
um conjunto de pré-condições em diversos operadores das classes
``Board'', ``Championship'' e ``Game''.

Primeiro, para facilitar a compreensão do código, foi criada a
seguinte constante na classe ``Board'':

\begin{vdm_al}
  values
  public attemptLength : nat1 = 4;
\end{vdm_al}

E foi também definida esta invariante sobre a variável ``attempts''
que impede que ela contenha sequências com um tamanho diferente do
valor que está em ``attemptLength'':

\begin{vdm_al}
  inv forall attempt in set elems attempts & len attempt = attemptLength and
    len attempts <= maxAttempts;
\end{vdm_al}

O mesmo foi feito para a variável ``solution'' que contêm o código a
quebrar do tabuleiro:

\begin{vdm_al}
  inv len solution = attemptLength;
\end{vdm_al}

Apesar de estas invariantes serem suficientes para cumprir com a
restrição indicada, a fim de se conseguir detetar este tipo de erros o
mais cedo possível, foram também criadas as seguintes pré-condições
nos construtores da classe ``Board'':

\begin{vdm_al}
  public Board : seq1 of Color`Color ==> Board
  Board (correctPlay) == ...
  pre len correctPlay = attemptLength
  post solution = correctPlay and
    attempts = [];
\end{vdm_al}

\begin{vdm_al}
  public Board : seq1 of Color`Color * nat1 ==> Board
  Board (correctPlay, maxNumberOfTries) == ...
  pre len correctPlay = attemptLength and
    maxNumberOfTries in set maxAttemptsAvailable
  post solution = correctPlay and
    maxAttempts in set maxAttemptsAvailable;
\end{vdm_al}

No caso do construtor ``Board()'' como este não recebe nenhuma solução
do utilizador esta é gerada aleatoriamente pelo computador então a
restrição foi colocada na pós-condição em vez de se meter na
pré-condição:

\begin{vdm_al}
  public Board : () ==> Board
  Board() == ...
  post len solution = attemptLength and attempts = [];
\end{vdm_al}

Esta restrição foi também colocada como pré-condição do operador
``makeAPlay'' da classe ``Board'', devido ao fato de a tentativa que
vai ser inserida no tabuleiro também precisar de ter um comprimento
igual a ``attemptLength''.

\begin{vdm_al}
  public makeAPlay : seq1 of Color`Color ==> ()
    makeAPlay (attempt) == ...
  pre len attempt = attemptLength and
    not isGameOver()
  post attempts = attempts~ ^ [attempt];
\end{vdm_al}

Foi também implementada esta restrição no operador ``addGame'' da
classe ``Championship'' de forma a impedir que o sistema crie jogos
cujas soluções tenham um comprimento diferente a ``attemptLength''.

\begin{vdm_al}
  public addGame : map String to (seq of Color`Color) ==> Game
  addGame(teamsSolutions) == ...
  pre dom teamsSolutions subset teams and
    dom teamsSolutions not in set oldMatchups() and
    forall solution in set rng teamsSolutions & len solution = Board`attemptLength
  post RESULT in set games;
\end{vdm_al}

E por fim, esta restrição foi implementada como uma pré-condição no construtor
``Game(teamsSolutions, championship)'' pois este construtor é
responsável por criar objetos da classe ``Board'' e assim impedimos o
sistema de tentar criar jogos que tivessem solução que quebrassem esta
restrição.

\begin{vdm_al}
  public Game : map String to seq1 of Color`Color * Championship ==> Game
  Game (teamsSolutions, championship) == ...
  pre card (dom teamsSolutions) = 2 and
    forall team in set dom teamsSolutions &
      team in set championship.getTeams() and
    forall solution in set rng teamsSolutions &
      len solution = Board`attemptLength

  post forall team in set (dom gameInstances) &
    let opponentTeam = dom teamsSolutions \ {team} in
      {gameInstances(team).getSolution()} =
      rng (opponentTeam <: teamsSolutions);
\end{vdm_al}


Para implementar a restrição 5, que indica que o número máximo de
tentativas que o \emph{codebreaker} pode efetuar têm que ser igual a 8, 10 ou
12, foi criado um conjunto na classe ``Board'' que guarda esses
valores.

\begin{vdm_al}
  public maxAttemptsAvailable : set of nat1 = {8, 10, 12};
\end{vdm_al}

Para guardar de fato o valor do número máximo de tentativas de cada
instância da classe ``Board'' foi criada uma variável ``maxAttempts''
que guarda esse valor. Esta variável tem por defeito o valor 12.

\begin{vdm_al}
  private maxAttempts : nat1 := 12;
\end{vdm_al}

Foi também definida uma invariante sobre essa variável que indica que
o valor dela têm que pertencer aos valores máximos permitidos pelas
regras do jogo.

\begin{vdm_al}
  inv maxAttempts in set maxAttemptsAvailable;
\end{vdm_al}

Tal como já foi anteriormente mostrado, a variável ``attempts'' também
contêm uma invariante que impede que o comprimento dela seja maior que
o valor de ``maxAttempts''.

\begin{vdm_al}
  inv forall attempt in set elems attempts & len attempt = attemptLength and
    len attempts <= maxAttempts;
\end{vdm_al}

Mais uma vez, apesar de estas duas invariantes serem suficientes para
garantir a restrição do tamanho do tabuleiro ser 8, 10 ou 12, foi
também adicionado código às pré-condições do construtor
``Board(correctPlay, maxNumberOfTries)'' e do operador ``makeAPlay''.

Esta restrição foi então colocada na pré-condição do construtor
``Board(correctPlay, maxNumberOfTries)'', forçando assim que o valor
de ``maxNumberOfTries'' sofra a mesma restrição de domínio que a da
variável ``maxAttempts''.

\begin{vdm_al}
  public Board : seq1 of Color`Color * nat1 ==> Board
  Board (correctPlay, maxNumberOfTries) == ...
  pre len correctPlay = attemptLength and
    maxNumberOfTries in set maxAttemptsAvailable
  post solution = correctPlay and
    maxAttempts in set maxAttemptsAvailable;
\end{vdm_al}

Por fim, esta restrição encontra-se também ativa no operador
``makeAPlay'' devido ao fato de este invocar a função ``isGameOver''
que impede que a jogada seja feita caso o tamanho de ``attempts'' seja
igual ao valor de ``maxAttempts''.

\begin{vdm_al}
  public makeAPlay : seq1 of Color`Color ==> ()
    makeAPlay (attempt) == ...
  pre len attempt = attemptLength and
    not isGameOver()
  post attempts = attempts~ ^ [attempt];
\end{vdm_al}

\begin{vdm_al}
  public isGameOver : () ==> bool
    isGameOver () == ...
  post RESULT = (isSolutionCracked() or len attempts = maxAttempts);
\end{vdm_al}

A restrição 6, que indica que o número de peças com a cor certa e no
sítio certo de uma tentativa não pode ser superior ao tamanho da
solução que está definida para o tabuleiro onde essa tentativa é
feita.

Isso é facilmente implementado através de uma pós-condição no operador
``calculateRightColorsInRightPlaces''.

\begin{vdm_al}
  public calculateRightColorsInRightPlaces : seq of Color`Color ==> nat
    calculateRightColorsInRightPlaces (attempt) == ...
  pre len attempt = len solution
  post RESULT <= len solution;
\end{vdm_al}

Da mesma forma, a restrição 7, que indica que o número de peças com a
cor certa, mas que estão no sítio errado, de uma tentativa não pode
ser superior ao tamanho da solução que está definida para o tabuleiro
onde essa tentativa é feita é implementada através de uma pós-condição
no operador ``calculateRightColorsInWrongPlaces''.

\begin{vdm_al}
  -- This formula is given in:
  -- http://mathworld.wolfram.com/Mastermind.html
  public calculateRightColorsInWrongPlaces : seq of Color`Color ==> nat
    calculateRightColorsInWrongPlaces (attempt) == ...
  pre len attempt = len solution
  post RESULT <= len solution;
\end{vdm_al}

A restrição número 8 que indica que o número de equipas inscritas num
campeonato tem que ser um número par igual ou superior a 2 está
definida como uma invariante sobre a variável ``teams'' da classe
``Championship''.

\begin{vdm_al}
  inv card teams >= and (card teams) mod 2 = 0;  
\end{vdm_al}

Foi também adicionada esta restrição na pré-condição do construtor
``Championship(participants)''.

\begin{vdm_al}
  public Championship: set of String ==> Championship
  Championship(participants) == ...
  pre card participants >= 2 and (card participants) mod 2 = 0
  post teams = participants;
\end{vdm_al}

Como existe também o operador ``addTeams'' que permite adicionar
equipas ao campeonato depois deste ter sido criado, a restrição também
teve que ser colocada nas pré-condições desse operador.

\begin{vdm_al}
  public addTeams : set of String ==> ()
  addTeams(newTeams) == ...
  pre (newTeams inter teams = {}) and (card newTeams) mod 2 = 0
  post teams = teams~ union newTeams;
\end{vdm_al}

Finalmente, a última restrição que indica que todas as equipas que
participam em jogos de campeonato têm que estar inscritas nesse
campeonato é implementada através de uma invariante sobre a variável
``games'' da classe ``Championship''.

\begin{vdm_al}
  inv forall game in set games & game.getParticipantTeams() subset teams;  
\end{vdm_al}

O operador ``getParticipantTeams'' está implementado na classe
``Game'' e devolve um conjunto que indica as equipas que estão a
participar nesse jogo.

\begin{vdm_al}
  public getParticipantTeams : () ==> set of String
  getParticipantTeams () == return dom gameInstances
  post RESULT = dom gameInstances;
\end{vdm_al}

Existe também na classe ``Championship'' dois operadores,
``addQuickGame'' e ``addGame'', que adicionam jogos à variável
``games'' de um campeonato.

Como tal, é necessário verificar antes de se adicionar esses jogos à
variável ``games'' se ambas as equipas pertencem ao conjunto de
equipas indicado pela variável ``teams'' do campeonato a que se quer
adicionar esse jogo.

\begin{vdm_al}
  public addQuickGame: String * String ==> Game
  addQuickGame(team1, team2) == ...
  pre {team1, team2} subset teams and
    {team1, team2} not in set oldMatchups()
  post RESULT in set games;

\end{vdm_al}

\begin{vdm_al}
  public addGame : map String to (seq of Color`Color) ==> Game
  addGame(teamsSolutions) == ...
  pre dom teamsSolutions subset teams and
    dom teamsSolutions not in set oldMatchups() and
    forall solution in set rng teamsSolutions & len solution = Board`attemptLength
  post RESULT in set games;
\end{vdm_al}

Foi também implementada esta restrição no construtor
``Game(teamsSolutions, championship)'', impedindo assim o sistema de
criar jogos com equipas que não participem no campeonato passado pelo
argumento ``championship''.

\begin{vdm_al}
  public Game : map String to seq1 of Color`Color * Championship ==> Game
  Game (teamsSolutions, championship) == ...

  pre card (dom teamsSolutions) = 2 and
    forall team in set dom teamsSolutions &
      team in set championship.getTeams() and
    forall solution in set rng teamsSolutions &
      len solution = Board`attemptLength

  post forall team in set (dom gameInstances) &
    let opponentTeam = dom teamsSolutions \ {team} in
      {gameInstances(team).getSolution()} =
      rng (opponentTeam <: teamsSolutions);
\end{vdm_al}

