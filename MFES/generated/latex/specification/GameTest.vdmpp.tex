\begin{vdmpp}[breaklines=true]
class GameTest
operations    
    public static main: () ==> ()
              main() ==
               (
                      new GameTest().runAllTests();
    );
(*@
\label{main:8}
@*)
  
 private runAllTests : () ==> ()
  runAllTests () == (
   test1();
   test2();
   test3();
(*@
\label{runAllTests:14}
@*)
   test4();
   test5();
   test6();
   test7();
   test8();
   test9();
   test10();
  );
 
 static public AssertTrue : bool ==> ()
(*@
\label{AssertTrue:24}
@*)
  AssertTrue(cond) == return
    pre cond;
    
  --Teste para criar um jogo player vs PC e fazer uma jogada com a solucao.
  private test1: () ==> ()
(*@
\label{test1:29}
@*)
  test1() ==
  (  
  dcl gameBot: Game := new Game(5,4,1);
  gameBot.makeAPlay(gameBot.getSolution());   
  AssertTrue(gameBot.isSolutionCracked() and gameBot.isGameOver());
  );
  
  --Teste para criar um jogo player vs player e fazera um jogada com a solucao definida pelo codeMaker.
(*@
\label{test2:37}
@*)
  private test2: () ==> ()
  test2() ==
  (  
   dcl gamePlayers: Game := new Game("bbbb","ze","tony",5,4,1);
  gamePlayers.makeAPlay("bbbb");   
  AssertTrue(gamePlayers.isSolutionCracked() and gamePlayers.isGameOver());
  );
  
(*@
\label{test3:45}
@*)
  --Teste para verificar se o numero de pontos ganhos e igual as tentativas.
  private test3: () ==> ()
  test3() ==
  (  
  dcl gamePlayers: Game := new Game("bbbb","ze","tony",5,4,1);
  gamePlayers.makeAPlay("bgbb"); 
  gamePlayers.makeAPlay("bbbb"); 
  gamePlayers.winGame();
  AssertTrue(gamePlayers.getCodeMaker().getPoints() = 2  and gamePlayers.isGameOver());
(*@
\label{test4:54}
@*)
  );
  --Teste para verificar a mudanca de tentativas.
  private test4: () ==> ()
  test4() ==
  (  
  dcl gamePlayers: Game := new Game("bbbb","ze","tony",5,4,1);  -- 5 attempts default
  gamePlayers.setMaxAttempts(2); 
  gamePlayers.makeAPlay("bgbb");
  AssertTrue(gamePlayers.numberOfTriesRemaining() = 1);

  gamePlayers.makeAPlay("bbbb"); --Solution
(*@
\label{test5:65}
@*)
  gamePlayers.winGame();
  AssertTrue(gamePlayers.isSolutionCracked() and gamePlayers.isGameOver());
  );
  --Teste para verificar varias jogadas do codeBreaker.
  private test5: () ==> ()
  test5() ==
  ( 
  dcl gamePlayers: Game := new Game("bbbb","ze","tony",5,4,1);
  gamePlayers.setMaxAttempts(10); 
  gamePlayers.makeAPlay("bgbb");
  gamePlayers.makeAPlay("bgbo");
  gamePlayers.makeAPlay("bgyb");
  gamePlayers.makeAPlay("rgbb");
  gamePlayers.makeAPlay("bgbo");
(*@
\label{test6:79}
@*)
  gamePlayers.makeAPlay("bgbm");
  gamePlayers.makeAPlay("bbbb"); --Solution
  AssertTrue(gamePlayers.isSolutionCracked() and gamePlayers.isGameOver());
  );
  -- Testar se ao fim de um jogo a funcao de cada jogador e trocada.
  private test6: () ==> ()
  test6() ==
  ( 
  dcl gamePlayers: Game := new Game("bbbb","ze","tony",5,4,2); 
  --codeMaker => Ze || codeBreaker => tony
  gamePlayers.changePlayer();
  AssertTrue(gamePlayers.getCodeBreaker().getName() = "ze" and gamePlayers.getCodeMaker().getName() = "tony" and gamePlayers.getMatches()=1);
  );
  -- Teste criar equipa
(*@
\label{test7:93}
@*)
  private test7: () ==> ()
  test7() ==
  ( 
    dcl p1:Player:= new Player("ze");
   dcl p2:Player:= new Player("tony");
   dcl t:Team:= new Team("FEUPinhos",p1,p2);  
  AssertTrue(t.getPlayer1().getName()="ze" and t.getPlayer2().getName()="tony" and t.getName() = "FEUPinhos");
  );
  -- Teste para calcular cores em sitio correcto e cores certas 
(*@
\label{test8:102}
@*)
  private test8: () ==> ()
  test8() ==
  ( 
  dcl tempGame :Game := new Game("baaa","ze","tony",5,4,1);
  AssertTrue(tempGame.calculateRightColorsInRightPlaces("baac")=3 and tempGame.calculateRightColorsInWrongPlaces("cccb")= 1);
  );
  -- Teste para torneio
(*@
\label{test9:109}
@*)
  private test9: () ==> ()
  test9() ==
  ( 
   dcl p1:Player:= new Player("ze");
   dcl p2:Player:= new Player("tony");
   dcl t1:Team:= new Team("FEUPinhos",p1,p2);
   dcl p3:Player:= new Player("to");
   dcl p4:Player:= new Player("rui");
   dcl t2:Team:= new Team("FEPinhos",p1,p2);
   dcl tor: Tournament := new Tournament({t1.getName() |-> t1, t2.getName() |-> t2},5,4);
   
   AssertTrue(tor.getTeamNames() = {"FEUPinhos", "FEPinhos"});
   AssertTrue(tor.getTeamPoints("FEUPinhos") = 0);
   
   tor.startGame ("bbbb",p1,p3);
   AssertTrue(tor.game.isSolutionCracked() = false);
  );
  --Teste para escrever torneio num ficheiro
  private test10: () ==> ()
(*@
\label{test10:128}
@*)
  test10() ==
  ( 
   dcl p1:Player:= new Player("ze");
   dcl p2:Player:= new Player("tony");
   dcl t1:Team:= new Team("FEUPinhos",p1,p2);
   dcl p3:Player:= new Player("to");
   dcl p4:Player:= new Player("rui");
   dcl t2:Team:= new Team("FEPinhos",p1,p2);
   dcl tor: Tournament := new Tournament({t1.getName() |-> t1, t2.getName() |-> t2},5,4);
   dcl io : IO := new IO();
   tor.startGame("bbbb",p1,p3);
   tor.game.makeAPlay("bcbc");
   tor.game.makeAPlay("cbcb");
   AssertTrue(io.fecho("torn.txt", tor.dumpTournamentToString(), <start>) = true);
  );
  
 traces 
  t1:test1();  
   t2:test2();
   t3:test3(); 
   t4:test4();          
   t5:test5();
   t6:test6();
   t7:test7();
   t8:test8();
   t9:test9();
   t10:test10();
            
end GameTest
\end{vdmpp}
